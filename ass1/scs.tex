\documentclass[12pt,a4paper]{scrartcl}

%\usepackage{algorithmic}    % Typesetting for pseudocode
%\usepackage{algorithm}      % Formatting for general algorithm blocks
\usepackage{fancyhdr}       % Gives fancy header
\usepackage{mdwlist}        % List related commands
\usepackage{url}            % Nicer URL formatting
\usepackage{new3151defs}    % COMP[39]151 defs
\usepackage{fancyvrb}

% Automata package
\usepackage{tikz}
\usetikzlibrary{arrows, automata, positioning, shapes}

% Page header
%\pagestyle{fancy}
%\lhead{COMP[39]151 Warmup Assignment
%\rhead{Timothy Wiley, z3109831}

% Line spacing 1.6 for 'double'
%\linespread{2}

% Declare commonly used graphic extensions and precedence
\DeclareGraphicsExtensions{.pdf,.png,.jpg}

\begin{document}

\title{COMP[39]151 Warm-up assignment}
\author{Aditya Keswani (z3242379) \\ 
        \texttt{akeswani@cse.unsw.edu.au} \\ 
        and \\ 
        Timothy Wiley (z3109831) \\
        \texttt{timothyw@cse.unsw.edu.au} }

\maketitle

This report is divided into ?? sections:
\begin{enumerate}
    \item A description of the implementation and justification for the choices made
    \item An explanation of the proof of the implementation in Promela
\end{enumerate}

\section{Description of Implementation}
The behaviour of the dancers and audience members in our implementation of the ``Dancing with the Shakes'' system is as follows:
\begin{itemize}
    \item Audience members request a dancer to watch and then wait for it to start dancing.
    \item Dancers greedily compete to get onto stage, i.e. if possible the dancer will choose itself to be on stage rather than deferring to another dancer.
    \item Dancers are only allowed to determine if they can go onto stage one at a time. 
    \item Dancers may only get onto stage if they have been requested or if there are no other requests that can be fulfilled at this time.
    \item Once a dancer is on stage, the audience members requesting it are notified and wait for the dance to finish
    \item Once the dancer has finished dancing, the audience members are again notified and can proceed to their vegetate phase.
    \item The dancer resets everything and then this process repeats.
\end{itemize}

\subsection{Threads}
We use two types of threads:
\begin{itemize}
    \item One thread (\texttt{audienceThread()}) per audience member.
          We do not distinguish between audience members watching live or at home on their TV.
    \item One thread (\texttt{dancerThread()()}) per dancer.
          We use the same code to simulate both aged and professional dancers.
\end{itemize}

\subsection{Shared Variables}
We use three types of shared variables.
\begin{itemize}
    \item Shared constants,
    \item Shared variables to communicate between the audience members and dancers, and
    \item Shared variables to communicate between dancers.
\end{itemize}

\subsubsection{Constants}
We use a set of shared variables that act as constants once the program has been initiated.
These variables track:
\begin{itemize}
    \item The number of dancers (\texttt{nDancers}) both aged (\texttt{nAgedDancers}) and pro (\texttt{nProDancers}),
    \item The number of audience members (\texttt{nAudience}), and
    \item The number of rounds for each audience member (\texttt{nRounds}).
\end{itemize}
Also for reference later, we index the \texttt{nDancers} dancers in the following way:
\begin{itemize}
    \item Indices $[0, \textrm{nAgedDancers})$ are all of the aged dancers, and
    \item Indices $[\textrm{nAgedDancers}, \textrm{nDancers})$ are all of the pro dancers.
\end{itemize}

\subsubsection{Audience to Dancer Variables}
The shared variables to communicate between the audience threads and dancer threads are:
\begin{itemize}
    \item \texttt{toWatch[]} -
          an array of size \texttt{nDancers} of integers that indicates the number of audience members wishing to watch each dancer.
    \item \texttt{validRequest[]} -
          an array of size \texttt{nDancers} of boolean values that indicates whether a request for a dancer is currently considered to be valid (i.e. follows the rules)
    \item \texttt{watchMutexes[]} -
          an array of size \texttt{nDancers} of mutexes that prevents audience members selecting a dancer while it is already dancing.
    \item \texttt{toWatchSemaphores[]} -
          an array of size \texttt{nDancers} of semaphores that audience members wait on until their dancer begins dancing and signals them.
    \item \texttt{nowWatchingSemaphore} -
          a semaphore that audience members wait on after their dancer begins dancing until it finishes dancing.
    \item \texttt{nAudienceWatching} -
          an integer that indicates the number of audience members yet to finish watching their dancer.
\end{itemize}

\subsubsection{Dancer to Dancer Variables}
The shared variables to communicate between the dancer threads are:
\begin{itemize}
    \item \texttt{dancerAged} and \texttt{dancerProOrAged}
          integers that indicate which dancers are currently on stage.
    \item \texttt{previousDancerAged} and \texttt{previousDancerProOrAged} - 
          integers that indicate the dancers that were previously on stage.
    \item \texttt{selectDancerMutex} -
          a mutex that prevents multiple dancers from simultaneously being set as one of the current dancers
    \item \texttt{leaveTogetherSemaphore} -
          a semaphore that forces the pro/aged dancer to wait for the aged dancer to finish resetting variables before proceeding
    \item \texttt{dancerProOrAgedDone} -
          an integer flag to ensure that the pro/aged dancer has finished dancing and also used to indicate when both dancers have left the stage
    \item \texttt{agedDancing} and \texttt{proOrAgedDancing} - 
          integer flags that are used to ensure that both dancers dance simultaneously
\end{itemize}

\subsection{Audience Member Threads}
The audience member threads are designed to be simple.
The actions of each audience thread is described in relation to the audience member states.
\begin{enumerate}
    \item In vegetation, no special action is taken. Only in this state may audience members die.
    \item Upon waking from vegetation, the audience member enters its selection and waiting stage. The audience member:
    \begin{enumerate}
        \item Selects a random dancer to watch,
        \item Then atomically:
        \begin{enumerate}
            \item Increments the dancer's \texttt{toWatch} value,
            \item Indicates that there is a request for the dancer using its \texttt{validRequest} value,
        \end{enumerate}
    \end{enumerate}
    \item The audience member then waits for its chosen dancer to appear on stage by waiting on a toWatchSemaphore.
          Once this semaphore is signalled the audience member enters its watching stage.
    \item The audience member waits for its chosen dancer to finish dancing by waiting on the nowWatchingSemaphore.
          Once the dancer leaves, the audience member returns to vegetation.
\end{enumerate}

\subsection{Dancer Threads}
The dancer threads are more complex as the process of selecting a dancer is involved.
Dancers iterate through the following stages:
\begin{enumerate}
    \item While either of the dancers have not been selected yet:
    \begin{enumerate}
        \item Check if there are any valid requests for dancers at the moment
        \item If there are no valid requests or if this dancer has been requested:
        \begin{enumerate}
            \item Check if this dancer is allowed to dance by checking the rules
            \item If so, set this dancer to one of the current dancers based on whether or not it is aged and which dancer has already been selected
            \item Regardless of whether it was selected, set validRequest to indicate there is no longer a valid request for this dancer
        \end{enumerate}
        \item If this dancer has become one of the current dancers:
        \begin{enumerate}
            \item Wait until the other dancer has also been selected
            \item Signal the toWatchSemaphore for this dancer to allow the audience members to begin watching
            \item Wait until the other dancer has also signalled its audience members
            \item Signal the nowWatchingSemaphore to tell the audience members the it has finished dancing
            \item Wait for these audience members to stop watching
            \item If this is the aged dancer:
            \begin{enumerate}
                \item Wait for the pro/aged dancer to finish dancing
                \item Reset all of the variables and set the previous dancers to the current aged and pro/aged dancers
                \item Signal the pro/aged dancer that it can leave
            \end{enumerate}
            \item If this is the pro/aged dancer:
            \begin{enumerate}
                \item Signal the aged dancer to indicate that it has finished dancing
                \item Wait for the aged dancing to finish resetting variables
            \end{enumerate}
            \item Leave
        \end{enumerate}
    \end{enumerate}
\end{enumerate}

\section{Sketch Proof}
We present a sketch proof that our model of the ``Dancing with the Shakes'' system follows all of the rules.

\begin{itemize}
    \item {\bf At any given time at most one show by precisely two dancers of which at least one is aged is happening on stage.}

To get onto the stage, a dancer must be the dancer stored in \texttt{dancerAged} or \texttt{dancerProOrAged}. This ensures that only these 2 dancers can get onto the stage. 

Furthermore, one dancer cannot be both dancerAged and dancerProOrAged because the code is structured such that after a dancer is selected, it cannot loop around again to also be selected as the other dancer until after it has finished dancing. This prevents there being only 1 dancer dancing on the stage.

Also, the first condition that a dancer must pass before it can set itself as a current dancer is that either \texttt{dancerAged} or \texttt{dancerProOrAged} is currently available, i.e. set to NO\_DANCER. Therefore, after the 2 dancers have already been selected, more dancers cannot become the current dancers and get onto the stage. The selectDancerMutex also surrounds all of the code for dancer selection, preventing multiple dancers being inside the selection code already when \texttt{dancerAged} or \texttt{dancerProOrAged} is being set.

Finally, as described earlier, the dancers whose \texttt{id} is less than \texttt{nAged} are considered aged dancers. The only dancers allowed to be set as the \texttt{dancerAged} dancer are those that have such an \texttt{id}. Thus, we ensure that one of the dancers is always aged.

    \item {\bf Nobody dances for two songs in a row.}

The previous dancers on stage are stored in two variables \texttt{previousAged} and \texttt{previousProOrAged}. These only change when dancers leave the stage. A dancer is not allowed to be selected as one of the current dancers if they are one of the dancers in these variables. Therefore, a dancer cannot be selected to dance on stage twice in a row.

    \item {\bf If audience member m requests dancer d while no other audience member is requesting or watching a conflicting dancer, then m should see d on stage/TV within a bounded number of his own steps (in the requesting phase)}

In our system, before a dancer can set itself as one of the current dancers there must either be no valid requests for other dancers or the dancer must itself be one of the requested dancers. Therefore, if m requests dancer d and there is nobody else requesting or watching a conflicting dancer, then no dancer other than d will be allowed to set itself as one of the current dancers. This is because every other dancer will see that there are valid requests but that they themselves have not been requested. Only dancer d will see that it has been requested and thus be allowed to set itself as one of the current dancers. After that, d will indicate that its request is no longer valid, allowing the remaining dancers to compete to be the other dancer. Thus, d will always get to be in the next pair to dance on stage, so m will get to see it on stage 

    \item {\bf If audience member m requests dancer d then m must eventually watch d dance}

As described earlier, the dancers compete to try and be selected as the current dancer. If there are requests for multiple dancers, including m's request for d, then these dancers will all compete and can all potentially become the current dancer. If we assume weak fairness, then eventually dancer d will get a chance to compete and until its request is fulfilled, it will keep trying to become the current dancer. Hence eventually, it will beat the other dancers, acquire the selectDancerMutex and because there is a request for it, it will become the current dancer (and will signal the audience member m to watch it).

\end{itemize}

\section{Performance}
In the dancer threads, the time taken to select the dancers depends upon how these threads are scheduled. If there are no dancers currently requested by audience members, then at most after all of the dancers have been scheduled once and had a chance to acquire the selectDancerMutex, there will be two dancers ready to dance on stage. If there are requests, then at worst we would have to go through all of the dancers once to fulfil one request and then go through them once more to find another dancer who is allowed/requested to dance. Thus, the time taken by this selection phase will be determined by the number of dancers and how long it takes for them all to be scheduled as described.

After the selection phase, the loops that occur while the dancer is dancing, finishing and leaving the stage iterate through the number of dancers and the number of audience members. Hence, the time taken for this part of the dancer threads is based on the sum of the number of dancers and the number of audience members.

Meanwhile, the audience spend most of their time waiting for the dancer that they request. Without these periods of waiting, these threads would simply complete in constant time.

The time spent on communication between the dancer and audience threads is based on the number of audience members. This is because this communication involves the dancers signalling all of the audience members currently waiting on them, and at maximum, all of the audience members can be waiting on one dancer, so the dancer will have to iterate through all of them. The current dancers communicate with each other using variables like aged/proDancing, leaveTogetherSemaphore and dancerProOrAgedDone, which are just constant time communication. The dancer threads also need to reset arrays like validRequest, which are the same size as the number of dancers, so there is also some communication that takes time based upon the number of dancers.

\section{Promela Proof}
To recap, the properties that we desire are:
\begin{enumerate}
    \item There are always two dancers on stage (unless dancers on stage are being selected).
    \item At least one dancer is aged.
    \item No dancer can dance on stage twice in a row.
    \item Eventual entry for audience members - that is they eventually witness the dancer they wish to see.
    \item Lack of unnecessary delay for audience members.
          That is they audience members within a bounded number of steps.
\end{enumerate}

These properties are given by a series of assertions and LTLs, and we use Promela to verify these.

\subsection{Dancer properties}
To verify the properties 1 - 3 in relation to dancers, we use a set of assertions at the ``dance'' point in the dancer code:
\begin{lstlisting}
assert(agedDancing == 1);
assert(proDancing == 1);
assert(dancerAged != NO_DANCER);
assert(dancerProOrAged != NO_DANCER);
assert(dancerAged != dancerProOrAged);
assert(dancerAged < N_AGED);
assert(dancerAged != previousAged);
assert(dancerAged != previousProOrAged);
assert(dancerProOrAged != previousAged);
assert(dancerProOrAged != previousProOrAged);
\end{lstlisting}

We also make use of the following LTLs
\begin{lstlisting}
ltl p3 { [] (dancerAged < N_AGED && dancerProOrAged < N_DANCERS) };
\end{lstlisting}


The dance point is where both dancers are on stage an dancing.
The first two assertions use the \texttt{agedDancing} and \texttt{proDancing} flags.
We also assert that the \texttt{dancerAged} and \texttt{dancerProOrAged} are valid.
We combine this with the LTL $p3$ to ensure the dancers must be within valid ranges.
Finally we assert the two dancers are different.
The combination of these assertions provides that at the point of dancing there are two dancers on stage, as required.
%TODO nOnStage counter

To assert that one dancer is aged, we use the \texttt{dancerAged} variable.
As aged dancers are in the range $[0,\textrm{N\_AGED}]$, this value being below \texttt{N\_AGED} gives us at least one aged dancer.
This property is also asserted by the LTL $p3$.
Thus one dancer must be aged, as required.

To ensure no dancer may dance twice in a row, we maintain the previous dancers when the current dancers leave the stage.
As this is the only place the dancers are modified, we know they must be the previous dancers.
We then assert the current dancers are not the same as the previous dancers.
This gives us that no dancer is on stage for two consecutive songs, as required.

\subsection{Eventual entry for Audience}
To give eventual entry we use the LTLs:
\begin{lstlisting}
ltl r0 { [] (audience@requestedDancer -> <> audience@watchingDancer) };
ltl r1 { [] (audienceTracked@requestedDancer -> <> audienceTracked@watchingDancer) };
\end{lstlisting}
We also label the two types of audience threads at the point where they request a dancer, and the point they watch the dancer.
The verification of this LTL means that any given audience member will see their desired dancer on stage, as required.

\subsection{Lack of unnecessary delay for audience}
The proof of this property in Promela is difficult, and may be incomplete.
The property requires that:
\begin{itemize}
    \item Other audience members are in their vegetation state, or
    \item watching a non-competing dancer.
\end{itemize}

To make the poof we introduce a second audience thread \texttt{audienceTracked} which is identical to the normal audience thread except:
\begin{itemize}
    \item It does not update the \texttt{audienceWaiting} counter, and
    \item It stores the dancer it selected in a shared variable.
\end{itemize}

We then introduce the LTL:
\begin{lstlisting}
ltl r3 { [] ((audienceTracked@requestedDancer && [](audienceWaiting == 0)) ->
         <> audienceTracked@watchingDancer) };

\end{lstlisting}

This LTL ensures that where \texttt{audienceTracked} is the only audience member selecting a dancer,
    and no other dancer can begin it's selection process, that eventually the tracked audience watches their dancer.
This is sufficient to show that there is a bound on the length of time required to wait for an audience member to see the dancer on stage.
If the watching state is not eventually reached, then there is no bound.
As it is reached, there is a bound as required.

\end{document}
