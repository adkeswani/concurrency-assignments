\documentclass[12pt,a4paper]{scrartcl}

%\usepackage{algorithmic}    % Typesetting for pseudocode
%\usepackage{algorithm}      % Formatting for general algorithm blocks
\usepackage{fancyhdr}       % Gives fancy header
\usepackage{mdwlist}        % List related commands
\usepackage{url}            % Nicer URL formatting
\usepackage{new3151defs}    % COMP[39]151 defs
\usepackage{fancyvrb}

% Automata package
\usepackage{tikz}
\usetikzlibrary{arrows, automata, positioning, shapes}

% Page header
%\pagestyle{fancy}
%\lhead{COMP[39]151 Warmup Assignment
%\rhead{Timothy Wiley, z3109831}

% Line spacing 1.6 for 'double'
%\linespread{2}

% Declare commonly used graphic extensions and precedence
\DeclareGraphicsExtensions{.pdf,.png,.jpg}

\begin{document}

\title{COMP[39]151 Warm-up assignment}
\author{Aditya Keswani (z3242379) \\ 
        \texttt{akeswani@cse.unsw.edu.au} \\ 
        and \\ 
        Timothy Wiley (z3109831) \\
        \texttt{timothyw@cse.unsw.edu.au} }

\maketitle

This report is divided into two sections:
\begin{enumerate}
    \item A description of the implementation and justification for the choices made
    \item An explanation of the proof of the implementation in Promela
\end{enumerate}

\section{Description of Implementation}
We have modelled the ``Dancing with the Shakes'' system using a set of threads, shared and private variables.

We use two types of threads:
\begin{itemize}
    \item One thread (\texttt{runAudience()}) per audience member. We do not distinguish between audience members watching live or at home on their TV.
    \item One thread (\texttt{runDancers)()}) to simulate the TV show selecting the next dancer and having dancers perform on stage.
\end{itemize}

In order to control the actions of audience members through their four stages we use a set of semaphores.
\begin{itemize}
    \item \texttt{toWatchSempahores[]} - an array of semaphores, with one for each dancer.
          Audience members wait on these semaphores when they have selected a dancer to watch and are waiting for the dancer to appear on stage
    \item \texttt{nowWatchingSemaphore} - a single semaphore on which audience members wait while the dancer they selected is performing on stage.
\end{itemize}
We have chosen to use semaphores over busy waits (ie. await statements) to control the actions of audience members.

We also use a number of shared variables to pass information between the audience members and TV show.
\begin{itemize}
    \item \texttt{toWatch[]} - an array of integers that records the number of audience members wishing to watch each dancer.
          We would prefer to use the value of the corresponding \texttt{toWatchSemaphores} value, however in pthreads semaphores do not become negative.
    \item \texttt{nWatching} - an integer to record the number of audience members currently actively watching the dancer on stage.
          This does not include audience members waiting for a dancer to appear on stage.
\end{itemize}

As C + pthreads does not have atomic control structures we also make use of a number of mutexs to control access to the shared variables.
\begin{itemize}
    \item \texttt{watchMutex} - controls access to the \texttt{toWatch} array.
          It also controls access to the \texttt{toWatchSempahores} array of semaphores.
    \item \texttt{nowWatchingMutex} - controls access to the \texttt{nWatching} integer.
\end{itemize}

\section{Promela Proof}

\end{document}
