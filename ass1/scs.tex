\documentclass[12pt,a4paper]{scrartcl}

%\usepackage{algorithmic}    % Typesetting for pseudocode
%\usepackage{algorithm}      % Formatting for general algorithm blocks
\usepackage{fancyhdr}       % Gives fancy header
\usepackage{mdwlist}        % List related commands
\usepackage{url}            % Nicer URL formatting
\usepackage{new3151defs}    % COMP[39]151 defs
\usepackage{fancyvrb}

% Automata package
\usepackage{tikz}
\usetikzlibrary{arrows, automata, positioning, shapes}

% Page header
%\pagestyle{fancy}
%\lhead{COMP[39]151 Warmup Assignment
%\rhead{Timothy Wiley, z3109831}

% Line spacing 1.6 for 'double'
%\linespread{2}

% Declare commonly used graphic extensions and precedence
\DeclareGraphicsExtensions{.pdf,.png,.jpg}

\begin{document}

\title{COMP[39]151 Warm-up assignment}
\author{Aditya Keswani (z3242379) \\ 
        \texttt{akeswani@cse.unsw.edu.au} \\ 
        and \\ 
        Timothy Wiley (z3109831) \\
        \texttt{timothyw@cse.unsw.edu.au} }

\maketitle

This report is divided into two sections:
\begin{enumerate}
    \item A description of the implementation and justification for the choices made
    \item An explanation of the proof of the implementation in Promela
\end{enumerate}

\section{Description of Implementation}
We have modelled the ``Dancing with the Shakes'' system using a set of threads, shared and private variables.

We use two types of threads:
\begin{itemize}
    \item One thread (\texttt{runAudience()}) per audience member. We do not distinguish between audience members watching live or at home on their TV.
    \item One thread (\texttt{runDancers)()}) to simulate the TV show selecting the next dancer and having dancers perform on stage.
\end{itemize}

In order to control the actions of audience members through their four stages we use a set of semaphores.
\begin{itemize}
    \item \texttt{toWatchSemaphores[]} - an array of semaphores, with one for each dancer.
          Audience members wait on these semaphores when they have selected a dancer to watch and are waiting for the dancer to appear on stage
    \item \texttt{nowWatchingSemaphore} - a single semaphore on which audience members wait while the dancer they selected is performing on stage.
\end{itemize}
We have chosen to use semaphores over busy waits (ie. await statements) to control the actions of audience members.

We also use a number of shared variables to pass information between the audience members and TV show.
\begin{itemize}
    \item \texttt{toWatch[]} - an array of integers that records the number of audience members wishing to watch each dancer.
          We would prefer to use the value of the corresponding \texttt{toWatchSemaphores} value, however in pthreads semaphores do not become negative.
    \item \texttt{nWatching} - an integer to record the number of audience members currently actively watching the dancer on stage.
          This does not include audience members waiting for a dancer to appear on stage.
\end{itemize}

As C + pthreads does not have atomic control structures we also make use of a number of mutexs to control access to the shared variables.
\begin{itemize}
    \item \texttt{watchMutex} - controls access to the \texttt{toWatch} array.
    \item \texttt{nowWatchingMutex} - controls access to the \texttt{nWatching} integer.
\end{itemize}

The actions of each thread are now discussed.

\subsection{Audience Member Threads}
The actions of each audience thread is described in relation to the audience member states.
\begin{enumerate}
    \item In vegetation, no special action is taken
    \item On wake from vegetation, the audience member enters their selection and waiting stage. They
    \begin{enumerate}
        \item Select a random dancer to watch,
        \item Take the \texttt{watchMutex},
        \item Increment the dancer's \texttt{toWatch} value,
        \item Release the mutex, and
        \item Wait on the dancers \texttt{toWatchSemaphores}
    \end{enumerate}
    \item On notification the audience member enters the watching state.
    \begin{enumerate}
        \item They increment \texttt{nWatching}, and
        \item Wait on the \texttt{nowWatchingSemaphore}
    \end{enumerate}
    \item On notification, the audience member has observed the dancer leave the stage. They decrement \texttt{nWatching} and return to vegetating.
\end{enumerate}

The use of semaphores ensures that audience members do not enter the active watching state until their dancer is on stage.
The semaphores also ensure that the audience member watches the dancer leave the stage before returning to their vegetative state

\subsection{TV Show Thread}
The TV show thread loops through the following actions:
\begin{enumerate}
    \item Select the new dancers,
    \item Record the number of audience members watch the dancers,
    \item Notify that number of audience members,
    \item Simulate the act of the dancers performing on state,
    \item Notify the audience watching the dancers that their performer has left the stage.
\end{enumerate}

The selection of the next dancers to perform is done as a linear search through all of the available dancers for one whose respective \texttt{toWatch} value is $>0$.
The available dancers are all who have not just left the stage.
If no such suitable dancer can be found (ie an available dancer an audience member wishes to watch), the first available dancer with their \texttt{toWatch = 0} is selected.
This selection process is perform twice.
The first time old aged dancers are considered, and the second time all dancers.

Once the dancers all waiting audience members for those dancers are notified.
The steps for this are:
\begin{enumerate}
    \item The thread waits for all audience members who were previously watching to be notified on \texttt{nowWatchingSemaphore}.
          This ensures the semaphore is 0 before any additional audience members are allowed to enter their watching stage.
    \item Both \texttt{watchMutex} and \texttt{nowWatchingMutex} are taken.
          Taking the \texttt{watchMutex} prevents any audience member from adding itself to the number of audience members wishing to watch the selected dancers.
          This means that the following step will ensure all currently waiting audience members for the given dancers are notified.
    \item The \texttt{toWatchSemaphores} for each selected dancer are notified by the number of in their respective \texttt{toWatch} arrays.
    \item The thread waits until all audience threads have received the notification.
    \item The mutexes are released.
\end{enumerate}

The taking of the mutexes outside of the notification of the \texttt{toWatchSemaphores} is necessary to give eventual entry.
This is because the mutexes prevent any additional audience members from adding themselves onto the \texttt{toWatchSemaphores} after the value of the respective \texttt{toWatch} value has been read.
This combined with the TV thread waiting means all waiting audience members are guaranteed to pass their semaphore and enter the watching state before the dancers appear on stage.

After this notification process the dancers take the stage and perform.
At the conclusion of this the TV show thread notifies the audience members their dancer has left the stage.
This is done by notifying the \texttt{nowWatchingSemaphore} for each audience member that is watching.

The promela proof further explains and shows how the necessary properties for the system are met.

\section{Promela Proof}

\end{document}
